\documentclass{article}

\usepackage[french]{babel}
\usepackage[utf8]{inputenc}
\usepackage[T1]{fontenc}
\usepackage[]{amsmath}
\usepackage{graphicx}
\usepackage{subcaption}
\usepackage{hyperref}

%%%%%%%%%%%%%%%% Lengths %%%%%%%%%%%%%%%%
\setlength{\textwidth}{15.5cm}
\setlength{\evensidemargin}{0.5cm}
\setlength{\oddsidemargin}{0.5cm}

%%%%%%%%%%%%%%%% Variables %%%%%%%%%%%%%%%%
\def\projet{4}
\def\titre{Systèmes d'équations non-linéaires / Méthode de Newton-Raphson}
\def\groupe{4}
\def\equipe{3}
\def\responsible{mrekik018}
\def\secretary{jeamartinez}
\def\others{scomiti, aguerouani, acattarin}

\begin{document}

%%%%%%%%%%%%%%%% Header %%%%%%%%%%%%%%%%
\noindent\begin{minipage}{0.98\textwidth}
  \vskip 0mm
  \noindent
  { \begin{tabular}{p{7.5cm}}
      {\bfseries \sffamily
        Projet \projet} \\ 
      {\itshape \titre}
    \end{tabular}}
  \hfill 
  \fbox{\begin{tabular}{l}
      {~\hfill \bfseries \sffamily Groupe \groupe\ - Equipe \equipe
        \hfill~} \\[2mm] 
      Responsable : \responsible \\
      Secrétaire : \secretary \\
      Codeurs : \others
    \end{tabular}}
  \vskip 4mm ~

  ~~~\parbox{0.95\textwidth}{\small \textit{The goal of this project is to program algorithms dedicated to the research of roots of systems of non-linear equations. The method promoted here is the Newton-Raphson algorithm, and the goal is to evaluate the assets and liabilities of such a solution. This shall be done by testing the method in different settings. In the following, you will be proposed a list of applications where this algorithm is necessary. You must program at least two of these applications, and then write a summary of your experiments as a conclusion. More applications are not synomyms of better work.} \sffamily }
  \vskip 1mm ~
\end{minipage}

%%%%%%%%%%%%%%%% Main part %%%%%%%%%%%%%%%%
\section{Résolveur Newton-Raphson}
\label{sec:newton}

\section{Applications}
\label{sec:appli}

\subsection{Calcul des points de Lagrange}
\label{ssec:lagrange}

\subsection{Equilibre électrostatique}
\label{ssec:electrostat}

\end{document}
