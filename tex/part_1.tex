\subsection {Question 1} 
Puisque $f(U+V)=0$ et étant donné que $f(U+V)$ est approximé par $f(U)+H(U) \times V$, on a :

\begin{equation}
  f(U)+H(U) \times V=0
\end{equation}

\subsection{Question 3}

Une manière simple de tester la fonction dans le cas où $f$ est une fonction de $\mathbb{R}$ dans $\mathbb{R}$ est de prendre $f(x)=x^2-4$ et $x_0=1.1$. On choisit également le nombre d'itération N et le seuil de convergence epsilon. On obtient alors le résultat suivant :

$x_{sol}=2.0$
